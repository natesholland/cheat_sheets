\documentclass[landscape]{article}
%ss[10pt,landscape]{article}
\usepackage{multicol}
\usepackage{calc}
\usepackage{ifthen}
\usepackage[landscape]{geometry}
\usepackage{amsmath,amsthm,amsfonts,amssymb}
\usepackage{color,graphicx,overpic}
\usepackage{wrapfig}
\usepackage{hyperref}
\usepackage[shortlabels]{enumitem}
\usepackage{enumerate}


\pdfinfo{
/Title (170_cheat_sheet.pdf)
/Creator (TeX)
/Producer (pdfTeX 1.40.0)
/Author (Nate Holland)
/Subject (CS170)
/Keywords (pdflatex, latex,pdftex,tex)}

% This sets page margins to .5 inch if using letter paper, and to 1cm
% if using A4 paper. (This probably isn't strictly necessary.)
% If using another size paper, use default 1cm margins.
\ifthenelse{\lengthtest { \paperwidth = 11in}}
    { \geometry{top=.3in,left=.3in,right=.3in,bottom=.3in} }
    {\ifthenelse{ \lengthtest{ \paperwidth = 297mm}}
        {\geometry{top=1cm,left=1cm,right=1cm,bottom=1cm} }
        {\geometry{top=1cm,left=1cm,right=1cm,bottom=1cm} }
    }

% Turn off header and footer
\pagestyle{empty}

% Redefine section commands to use less space
\makeatletter
\renewcommand{\section}{\@startsection{section}{1}{0mm}%
                            {-1ex plus -.5ex minus -.2ex}%
                            {0.5ex plus .2ex}%x
                            {\normalfont\large\bfseries}}
\renewcommand{\subsection}{\@startsection{subsection}{2}{0mm}%
                            {-1explus -.5ex minus -.2ex}%
                            {0.5ex plus .2ex}%
                            {\normalfont\normalsize\bfseries}}
\renewcommand{\subsubsection}{\@startsection{subsubsection}{3}{0mm}%
                            {-1ex plus -.5ex minus -.2ex}%
                            {1ex plus .2ex}%
                            {\normalfont\small\bfseries}}
\makeatother

% Define BibTeX command
\def\BibTeX{{\rm B\kern-.05em{\sc i\kern-.025em b}\kern-.08em
    T\kern-.1667em\lower.7ex\hbox{E}\kern-.125emX}}

% Don't print section numbers
\setcounter{secnumdepth}{0}


\setlength{\parindent}{0pt}
\setlength{\parskip}{0pt plus 0.5ex}

%My Environments
\newtheorem{example}[section]{Example}
% -----------------------------------------------------------------------

\def\ci{\perp\!\!\!\perp}



\begin{document}
\raggedright
\footnotesize
\begin{multicols}{3}


% multicol parameters
% These lengths are set only within the two main columns
%\setlength{\columnseprule}{0.25pt}
\setlength{\premulticols}{1pt}
\setlength{\postmulticols}{1pt}
\setlength{\multicolsep}{1pt}
\setlength{\columnsep}{2pt}

\begin{center}
    \Large{\underline{CS 170 Cheat Sheet}} \\
\end{center}

\subsection*{Big O notation}

$ f, g \in \mathbb{N}$, $f = O(g)$ means that f grows no faster than g if $\exists c > 0$ s.t. $F(n) \leq c g(n)$ 

$ f = \Theta(g)$ means $g = O(f)$

$f = \Theta(g)$ IFF $ f = O(g) \ \& \ g = \Theta(g)$


\subsection*{Master Theorem}
Given: $T(n) = a \times T(\frac{n}{b}) + O(n^d)$

\begin{description}

\item[a)]
$O(n^d)$ if $d > log_b(a)$

\item[b)]
$O(n^dlog(n))$ if $d = log_b(a)$

\item[c)]
$O(n^{log_b(a)})$ if $d < log_b(a)$

\end{description}

\subsection*{Graph Algorithms}

\textbf{DFS: O(V + E)}

Guaranteed to visit every node reachable by $v$ before returning from $v$.
Can create topological sort of DAG.

\includegraphics[scale=0.5]{DFS}

\textbf{BFS: O(V + E)}

Used to find shortest path through an unweighted graph.

\includegraphics[scale=0.5]{BFS}

\textbf{Dijkstras: O((V + E) log V)}

Like BFS but with priority queue, used to find shortest path between two nodes on a weighted graph.

\includegraphics[scale=0.5]{dijkstra}

\textbf{Bellman Ford: O((V  E)}

Find shortest paths with negative edges as long as there are no negative cycles. Runs $V -1$ updates on all E edges.

\includegraphics[scale=0.5]{BF1}

\includegraphics[scale=0.5]{BF2}

\textbf{Kruskal: O((E log(V))}

Use the disjoint set trees to add edges in ascending order that don't complete a cycle. Used to find MST.

\includegraphics[scale=0.5]{kruskal}

\textbf{Prim's: O((E log(V))}

On each iteration, the subtree defined by X grows by one edge, namely, the lightest edge
between a vertex in S and a vertex outside S

\includegraphics[scale=0.42]{prim}

\subsection{FFT}

It is a black box which represents 2 polynomials as a list of points and then multiplies them together to create a new polynomial. Takes \textbf{O(N log N)} time. Uses roots of unity to determine where to multiply two polynomials together.

\textbf{$N^{th}$ Roots of Unity} can be found by: $ cos(\frac{2\pi j}{n}) + i \cdot sin(\frac{2\pi j}{n})$

\subsection{P/NP Basics}

\textbf{P:} search problem that can be solved in polynomial time.

\textbf{NP:} search problem that can be checked to be correct in polynomial time.

\textbf{NP complete:} search problem that is at least as hard as every other NP complete problem. Basically it reduces to circuit sat, could possibly be solved in polynomial time but we doubt it.

\textbf{NP hard:} there exists NP complete Y such that Y is reducible to X but can't go the other way around. Not actually in NP.

\subsection{Reductions}
A search problem is NP-complete if all other search problems reduce to it.

To reduce X to Y means to find a solution for X using Y.

\includegraphics[scale=0.42]{reduction}



\rule{0.3\linewidth}{0.25pt}
\newpage
\scriptsize
\bibliographystyle{abstract}
\bibliography{refFile}
\end{multicols}
\end{document}