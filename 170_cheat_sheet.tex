\documentclass[3pt,landscape]{article}
%ss[10pt,landscape]{article}
\usepackage{multicol}
\usepackage{calc}
\usepackage{ifthen}
\usepackage[landscape]{geometry}
\usepackage{amsmath,amsthm,amsfonts,amssymb}
\usepackage{color,graphicx,overpic}
\usepackage{wrapfig}
\usepackage{hyperref}
\usepackage[shortlabels]{enumitem}
\usepackage{enumerate}


\pdfinfo{
/Title (170_cheat_sheet.pdf)
/Creator (TeX)
/Producer (pdfTeX 1.40.0)
/Author (Nate Holland)
/Subject (CS170)
/Keywords (pdflatex, latex,pdftex,tex)}

% This sets page margins to .5 inch if using letter paper, and to 1cm
% if using A4 paper. (This probably isn't strictly necessary.)
% If using another size paper, use default 1cm margins.
\ifthenelse{\lengthtest { \paperwidth = 11in}}
    { \geometry{top=.3in,left=.3in,right=.3in,bottom=.3in} }
    {\ifthenelse{ \lengthtest{ \paperwidth = 297mm}}
        {\geometry{top=1cm,left=1cm,right=1cm,bottom=1cm} }
        {\geometry{top=1cm,left=1cm,right=1cm,bottom=1cm} }
    }

% Turn off header and footer
\pagestyle{empty}

% Redefine section commands to use less space
\makeatletter
\renewcommand{\section}{\@startsection{section}{1}{0mm}%
                            {-1ex plus -.5ex minus -.2ex}%
                            {0.5ex plus .2ex}%x
                            {\normalfont\large\bfseries}}
\renewcommand{\subsection}{\@startsection{subsection}{2}{0mm}%
                            {-1explus -.5ex minus -.2ex}%
                            {0.5ex plus .2ex}%
                            {\normalfont\normalsize\bfseries}}
\renewcommand{\subsubsection}{\@startsection{subsubsection}{3}{0mm}%
                            {-1ex plus -.5ex minus -.2ex}%
                            {1ex plus .2ex}%
                            {\normalfont\small\bfseries}}
\makeatother

% Define BibTeX command
\def\BibTeX{{\rm B\kern-.05em{\sc i\kern-.025em b}\kern-.08em
    T\kern-.1667em\lower.7ex\hbox{E}\kern-.125emX}}

% Don't print section numbers
\setcounter{secnumdepth}{0}


\setlength{\parindent}{0pt}
\setlength{\parskip}{0pt plus 0.5ex}

%My Environments
\newtheorem{example}[section]{Example}
% -----------------------------------------------------------------------

\def\ci{\perp\!\!\!\perp}



\begin{document}
\raggedright
\footnotesize
\begin{multicols}{3}


% multicol parameters
% These lengths are set only within the two main columns
%\setlength{\columnseprule}{0.25pt}
\setlength{\premulticols}{1pt}
\setlength{\postmulticols}{1pt}
\setlength{\multicolsep}{1pt}
\setlength{\columnsep}{2pt}

\begin{center}
    \Large{\underline{CS 170 Cheat Sheet}} \\
\end{center}

\subsection*{Big O notation}

$ f, g \in \mathbb{N}$, $f = O(g)$ means that f grows no faster than g if $\exists c > 0$ s.t. $F(n) \leq c g(n)$ 

$ f = \Theta(g)$ means $g = O(f)$

$f = \Theta(g)$ IFF $ f = O(g) \ \& \ g = \Theta(g)$


\subsection*{Master Theorem}
Given: $T(n) = a \times T(\frac{n}{b}) + O(n^d)$

\begin{description}

\item[a)]
$O(n^d)$ if $d > log_b(a)$

\item[b)]
$O(n^dlog(n))$ if $d = log_b(a)$

\item[c)]
$O(n^{log_b(a)})$ if $d < log_b(a)$

\end{description}

\subsection*{Graph Algorithms}

\textbf{DFS: O(V + E)}

Guaranteed to visit every node reachable by $v$ before returning from $v$.
Can create topological sort of DAG.

\textbf{BFS: O(V + E)}

Used to find shortest path through an unweighted graph.

\textbf{Dijkstras: O((V + E) log V)}

Like BFS but with priority queue, used to find shortest path between two nodes on a weighted graph.

\textbf{Bellman Ford: O((V  E)}

Find shortest paths with negative edges as long as there are no negative cycles. Runs $V -1$ updates on all E edges.

\textbf{Kruskal: O((E log(V))}

Use the disjoint set trees to add edges in ascending order that don't complete a cycle. Used to find MST.

\subsection{FFT}

It is a black box which represents 2 polynomials as a list of points and then multiplies them together to create a new polynomial. Takes \textbf{O(N log N)} time. Uses roots of unity to determine where to multiply two polynomials together.

$N^{th}$ Roots of Unity can be found by: $ cos(\frac{2\pi j}{n}) + i \cdot sin(\frac{2\pi j}{n})$

\rule{0.3\linewidth}{0.25pt}
\newpage
\scriptsize
\bibliographystyle{abstract}
\bibliography{refFile}
\end{multicols}
\end{document}